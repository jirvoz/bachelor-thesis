\chapter{Introduction}
Performance of operating system is crucial, because it can significantly affect all the applications running above it.
When a regression in new version occurs, business applications can be even financially affected.

Scheduler was simple. Multi-core CPUs made the scheduler little more complex, but those complications were tuned through time.
Next milestone was introduction of multi-CPU machines with separated memory for each physical unit.

Compared to functional testing, performance isn't evaluated as true/false result, but relative change to previous measurement.
Due to this complexity is hard to use common tools for inspecting performance.
The biggest regression in scheduler doesn't hide in uneffective code, but in wrong placement of processes to cores and it's queues.

The usual testing method is simulating load simmilar to real usage.
To achieve this there are many benchmarks targetting different types of load, usually many parallel process, sometimes communicating between each other.

Good performance tests usually create great amount of data, which is required to be processed, aggregaed and visualised.

\chapter{Linux process scheduling}
Scheduler is part of operating system which assigns processor time to tasks.
It's main goal is to maximize effectivity of CPU usage and fairness of assigned
CPU time to each task.

There are two opposing targets for scheduler - maximazing throughput, or
latency. While user's workstation aims for low response time, computational
servers require high throughput. Scheduler can be then tuned to fit the intended
purpose.

\section{Completely Fair Scheduler}
Completely Fair Scheduler is current Linux process scheduler, merged into version
2.6.23 of Linux kernel in 2007. Its author is Ingo Molnár, author of previous
O(1) scheduler as well.

It features queuing tasks in red-black tree structure ordered by the time spent
running on CPU so far. When the scheduler needs to choose next task to run, it
takes the leftmost node with lowest execution time. On multi-core systems it
keeps own queue for each core, which must be balanced for efficient utilization
of the system.

Time complexity of CFS scheduling is O(log N). Taking the leftmost node with
next task can be done in constant time, but queuing the task again requires
O(log N) operations to insert it back to the red-black tree.

Single processor is simple, problems come with NUMA nodes. Switching between
nodes or reading from another is expensive.
Another problem is when scheduler leaves cores idle, even where there are
runnable threads waiting.


\chapter{Performance measurement}
A basic way to measure performance of system is benchmark. They return some numbers representing throughput of system, but can't tell more about the system behavior.
To get a better insight on the system, there are many performance analysis tools to observe cpu and memory utilization, location of processes on cores or their time out of cpu.
\texttt{http://www.brendangregg.com/activebenchmarking.html}

\section{Benchmarks}

\subsection{NAS Parallel Benchmarks}
NAS Parallel Benchmarks is set of benchmarks focused on performance of highly
parallel computations on supercomputers. In addition to floating point
computations it targets communication and data movement between computation
nodes. Performed algorithms are based on large scale computational fluid
dynamics at the Numerical Aerodynamic Simulation (NAS) Program which is based at
NASA Ames Research Center.

Benchmarks are written in Fortran-90 or C language, beacause they were the most
commonly used programming languages in scientific parallel computing community
by the time they were created. They can be compiled with different classes of
problem sizes to suit machines with different amount of memory and computational
power.

Many parameters can be passed to benchmark also before execution to tweak
properties of behavior. One of them is number of computation thread, which in
lower amount slowers the run time, but allows to measure behavior of system
without full usage.

\begin{figure}
  \centering
  \includegraphics[width=10cm]{obrazky-figures/nas}
  \caption{Example of Scalar Penta-diagonal solver results from NAS Parallel with
  different number of computational threads.}
\end{figure}

\subsection{SPECjbb2005}
Java Business Benchmark behaves as a server-side Java application. It is primarily focused on measuring performance of Java implementation, but also reflects performance of operating system and cpu itself.
It models system of wholesale company as a multitier application. The benchmark
itself creates the load, measures the throughput and also generates simple
report in HTML and raw text formats.

The main output value is \emph{throughput} in units called \emph{SPECjbb2005
  bops} \footnote{Business operations per second}.
In case of using more JVM instances, there is a second unit called SPECjbb2005
bops/JVM representing average throughput of single JVM\footnote{Java virtual machine} instance.
Another collected metric is memory comsumption, which isn't that useful in
scheduler performance monitoring.

\subsection{LINPACK benchmark}
LINPACK Benchmark comes from LINPACK package, wich was used to solve systems of linear equations in single or double precision.

\subsection{Stream benchmark}

\section{Performance analysis tools}

\subsection{time}
Simple command, that can measure resource usage. Usually it's used to measure
running time of process.

It isn't bash builtin, but programm, that must be usually called from /usr/bin/time.

\subsection{ps}
This command lists current processes with information about them. In
\texttt{PSR} and \texttt{NUMA} column are useful informations of CPU and NUMA
node number, where is the process running.

\subsection{mpstat}

\subsection{turbostat}
This tool provides measuring of CPU usage and mainly its frequency.

\subsection{perf}


\chapter{Storing the results}
Benchmarks usually generate long human-readable output in text or even HTML
format. This is useful when analyzing single report. In the output are details
of the test run itself, simple resource usage or success of result validation.
However, the amount of result starts to rise with repeated runs, different
amount of instances and new versions kernels.

For the comparison of performance results, it is usually enough the number
representing throughput or time of each benchmark run. Those numbers can be
preprocessed from the benchmark output files to a format more suitable for quick
accesing required data.

\section{XML files}
XML is markup language, that can store heterogeneous data in tree structure.
However, the main feature of this format is human-readability. Even with the
inefficient storage of numerical data and repetition of tag and attribute names,
it allows user to quicky see the numbers without any processing of the data.

\section{Database}


\chapter{Displaying the results}

\section{Heatmaps}

\section{Boxplots}
Boxplot is method for displaying statistical properties of data groups. The
usually displayed values are minimum, 1\textsuperscript{st} quartile, median,
3\textsuperscript{rd} quartile and maximum.
This feature is useful to visualize accuaracy and reliability of measurement.
Then it's easier to distinguish performance regression from noise.

\begin{figure}
  \centering
  \includegraphics[width=12cm]{obrazky-figures/boxplot}
  \caption{Example of boxplot showing Mops of NAS benchmark with different
    number of threads}
\end{figure}


\chapter{Timelines}
An common way to get performance report is to compare two results. Usually, they
are called baseline and target results. The comparison of two results allows to
write many details about the measurement and changes between versions. Those
details usually contain clues to the cause of possible performance change.

However, sometimes is not enough to compare just two versions and larger amount
of results over longer period of time can bring a new perspective. There is much
more visible difference between the deviation from measurement error and the
performance change. It is also easier to find the versions, where a
performance degradation occured and where was fixed.

With larger amount of data, there can also emerge performance drops, which
appeared continuously over longer period of time and couldn't be spotted,
because they were in tolerance due to deviation.

\section{Result storage}
Benchmark results are stored on filesystem in directories. Each result consists
of two XML files with information and preprocessed main data from the test run,
more files with larger complementary data from analysis tools and compressed
original outputs from benchmark. First XML file contains metadata from the test
run including time, machine hostname, kernel and OS version, benchmark name,
configuration of environment, which could affect the result and few other data.
The second XML file contains the important preprocessed data itself.

The program has to go throgh all these results to choose the right ones for the
following creation of graphs.

\section{Comparison rules}
For automatic report generation are essential rules, which will specify results,
that can be used and in which role. I chose \emph{regular expressions} to match
properties of results. Regular expressions offer broad possibilities to describe
shape of kernel version or just value of any evironment configuration. To filter
all builds of kernel 4.18 works simple pattern \texttt{kernel-4.18\textbackslash
..*}.

To store the rules are stored in XML file with same node naming as in XML file
with result properties. The first level of XML document contains three nodes
representing purpose of rules.
\begin{itemize}
  \item \textbf{Baseline rules} specify the first result in plotted set. Acts as
    main result others are compared to.
  \item \textbf{Target rules} define the results to be plotted.
  \item \textbf{Starting rules} are for case, when base result is not from
    target set of results and specifying first target result with regex would be
    hard.
\end{itemize}
Each of theese nodes then contain in them nodes with the rules.


\chapter{Automatic evaluation}
With every expansion of regular kernel testing rises the amount of produced
results. With more machines with different configurations, benchmarks with
different focus or baselines from different supported versions, number of results
can rise with every new kernel version even by hundrets. This tends to automate
the repetitive classification of results between pass and fail to leave results
without significant change and focus on the ones with performance regressions.

\section{Marking of results}
To teach automatic classificator is essential large amount of data. To reduce
time spent on marking of the passed and failed results I created a form in
report to quicken this process.

On machine for storing results is running simple server application for
commiting requests from the form. The server is written in Python built on top
of Flask framework. On request it finds given result and modifies its file with
marking for learning.

\begin{figure}
  \centering
  \includegraphics[width=12cm]{obrazky-figures/teaching_table}
  \caption{Form from HTML comparison report page to mark data for machine learning}
\end{figure}


\chapter{Future work}


\chapter{Notes}
\begin{verbatim}
Linux scheduler
    CFS
    numa planning
        not easy for scheduler
        manual pinning option
            numactl
        group imbalance bug
    tune profiles
        focus on throughput or latency
Performance
    performance isn't simple pass/fail
    comparison of baseline and target kernel
    running just benchmark is waste of testing potential, it's good to collect more data
        http://www.brendangregg.com/activebenchmarking.html
    provisioning of machines with Beaker
    collecting system load data
        mpstat
            usage of every cpu
        numastat
        numatop
            usage of cpu and memory on each node
        ps
            psr and numa column
            cpu time
        time
            total, user and system time spent
        turbostat
        perf stat
        lstopo for hardware schema image
        free
    benchmarks
        specjbb2005
        specjvm2008
        nas parallel
        linpack benchmark
        stream (memory throughput)
        specjbb2015 - not stable
        hackbench - not tried yet
    scenarios
        variable number of benchmark instances
        with and without pinning processes to specific numa nodes
    result storing
        preprocessing to xml
        storing to database
        aggregating results
        computing statistical data
    plotting
        lstopo for plotting hardware topology
        bargraphs to show inaccuaracy
            operations per second (runtime of benchmark)
            time out of cpu
        detailed comparsion of two results
        timeline with many results over longer period of time
        heatmaps for core usage over time
            http://www.brendangregg.com/HeatMaps/utilization.html
        graph of process migration between numa nodes
Timelines
    analysis
    used technologies
    desing
    implementation
    output
Automatic regression detection
    motivation
    marking data
    preprocessing for classificator
    methods
        linear logistic regression
        decision trees
    evaluation
\end{verbatim}

\chapter{Specification}
\begin{enumerate}
\item Get acquainted with the existing methods for measuring performance of the Linux kernel scheduler and with means of storing of benchmarks results for further processing.
\item Study possible ways of processing these results with a focus on graphic interpretation and on methods for detection of performance degradation.
\item Design and implement a method for efficient graphic interpretation of long-term measurements.
\item Design and implement a method for automatic detection of performance regression.
\item Demonstrate the functionality of your implementation on at least two versions of the Linux kernel.
\item Evaluate the obtained results and discuss possibilities of further development of the project, especially of the automatic detection of performance regression.
\end{enumerate}

Literature:
\begin{itemize}
\item Lozi, Jean-Pierre, et al. "The Linux scheduler: a decade of wasted cores." Proceedings of the Eleventh European Conference on Computer Systems. ACM, 2016.
\item Daniel, P., and Cesati Marco. "Understanding the Linux kernel." (2007).
\item Bailey, David H., et al. "The NAS parallel benchmarks." The International Journal of Supercomputing Applications 5.3 (1991): 63-73.
\end{itemize}
