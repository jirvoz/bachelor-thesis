% Tento soubor nahraďte vlastním souborem s přílohami (nadpisy níže jsou pouze pro příklad)
% This file should be replaced with your file with an appendices (headings below are examples only)

% Umístění obsahu paměťového média do příloh je vhodné konzultovat s vedoucím
% Placing of table of contents of the memory media here should be consulted with a supervisor
%\chapter{Obsah přiloženého paměťového média}

%\chapter{Manuál} % manual

%\chapter{Konfigurační soubor} % Configuration file

%\chapter{RelaxNG Schéma konfiguračního souboru} % Scheme of RelaxNG configuration file

%\chapter{Plakát} % poster

\chapter{How to use this template}
\label{how}

This chapter describes the individual parts of the template, which is followed by instructions on how to use this template.

This version of the template is only temporary. The new one is going to be published by the end of the year 2017 and it will contain new instructions on how to properly use the template, obligatory instructions on how to work on bachelor's and master's thesis (recapitulation of instructions which are available online) and non-obligatory recommendations by some from the selected supervisors that are already available on-line (see links in literature file). The only files that will contain changes in the new version are \texttt{projekt-01-kapitoly-chapters.tex} and \texttt{projekt-30-prilohy-appendices.tex}. Content of each will be removed by every student and its content will be replaced by their own. The template is thus ready to be used even in the current version.

\section*{Description of the various parts of the template}

After you unpack the template, you will find the following files and directories:
\begin{DESCRIPTION}
  \item [bib-styles] Bibliography styles (see below). 
  \item [obrazky-figures] Directory for your images. It contains placeholder.pdf (i.e. TODO image, which can be helpful when working on the technical report). This image should not be present in the final solution. It is advised to shorten the directory name so it is only in the chosen language.
  \item [template-fig] Template figures (BUT logo).
  \item [fitthesis.cls] Template (style definitions).
  \item [Makefile] Makefile for compilation, count of regular pages, packing and so on (see below).
  \item [projekt-01-kapitoly-chapters.tex] File for your text (replace the content).
  \item [projekt-20-literatura-bibliography.bib] List of bibliography references (see below).
  \item [projekt-30-prilohy-appendices.tex] File for appendices (replace the content).
  \item [projekt.tex] The main thesis file -- definition of formal parts.
\end{DESCRIPTION}

The default bibliography style (czechiso) is a work of Ing. Martínek, while the English version (englishiso) is his translation with minor adjustments. There are some differences compared to the standard norm, yet it has been accepted at FIT BUT for a long time now. Alternatively, you can use the style of Ing. Radim Loskot or the style of Ing. Radek Pyšný\footnote{Bachelor thesis of Ing. Radek Pyšny \url{http://www.fit.vutbr.cz/study/DP/BP.php?id=7848}}. The alternative styles contain various improvements, however, they have not yet been properly tested. They can be considered as a beta version for those who aim for detailed and exemplary results and do not hesitate to study on how to properly format citations so the typeset results could be properly verified.

Apart from compiling to PDF, Makefile also offers:
\begin{itemize}
  \item renaming of files (see below),
  \item counting of regular pages,
  \item completing text with nonbreakable spaces,
  \item automated packing of files so the result can be submitted to supervisor for inspection (make sure all files are being packed, rectify otherwise).
\end{itemize}

It is not guaranteed that all necessary nonbreakable spaces would be added. It is always needed to manually check whether all line endings comply with the norm -- see Internet language guide\footnote{Internet language guide \url{http://prirucka.ujc.cas.cz/?id=880}}.

\paragraph {Watch out for page numbering!} In case the table of contents stretches for 2 pages and the second page contains only the \uv{Attachments} and~\uv{List of attachments} (but there are none of attachments), all pages would be incorrectly numbered about 1 (table of contents \uv{does not match the correct page numbers}). The same happens when there is only the \uv{Literature} section on the second or third page of the table of contents and it is also possible, that this problem occurs even in some other cases. The solution for this might be changing the table of contents or tweaking the page numbering settings or some other more sophisticated methods. \textbf{Therefore always check whether the page numbering is correct!}


\section*{Recommended workflow with the template}

\begin{enumerate}
  \item \textbf{Check whether you have the current version of a template.} In case you have some older version template, you can find the newest version, which contains updated information, fixed errors, etc., on the faculty's web page.
  \item \textbf{Choose the language}, in which you are going to write a technical report (Czech, Slovak or English) and consult your choice with your supervisor (unless you have already agreed on that). In case you choose other language than Czech, set the relevant template parameter in a file projekt.tex (e.g. \verb|documentclass[english]{fitthesis}| and translate the declaration and acknowledgements sections into English or Slovak language.
  \item \textbf{Rename the files.} You can find projekt.tex file in the template directory after unpacking. If you would have compiled this file, you would get a PDF containing a technical report named projekt.pdf. In case supervisors obtain many files named projekt.tex, they have to rename all of them. Therefore, it is recommended to rename this file so it contains your login and (appositely shortened) title of the work. When naming your file, avoid spaces, diacritics and special characters. Thus suitable name for your file may be e.g. uv{xlogin00-Information-extraction.tex}. To rename the file you can also use the included Makefile.
\begin{verbatim}
make rename NAME=xlogin00-Information-extraction
\end{verbatim}
  \item Fill in the required entries in the file which was originally named projekt.tex. That means the type of the work, year (year of handing in), title, your name, department (according to assignment), titles and the name of your supervisor, abstract, keywords and other formalities.
  \item Extended abstract in Czech can be enabled in the template by uncommenting the particular lines of code in \tt fitthesis.cls\rm .
  \item Replace the content of files with chapters, references and attachments with the content of your technical report. It may be convenient to place the individual attachments or chapters in separate files -- in case you decide for this option, it is recommended to follow the convention for file names by stating the name of a chapter after a number.
  \item Unless you need attachments, comment out the relevant part in the file projekt.tex and empty or delete this file. Do not include useless attachments with the only purpose to fill in a given file. A suitable attachment may be eg. a table of content of the attached storage media.
  \item Save the assignment downloaded in PDF from FIT IS in a file zadani.pdf and enable its insertion into the work with a template parameter in projekt.tex (\verb|documentclass[zadani]{fitthesis}|).
  \item In case you do not want to print colored links (i.e. red table of content -- not recommended unless consulted with a supervisor), you are going to create another PDF with a template parameter set for print: (\verb|documentclass[zadani,print]{fitthesis}|). The colored logo must not be printed black-and-white!
  \item You can generate a template for the cover, in which the work is going to be bound, in the faculty's information system near your assignment. Regarding the dissertation thesis, you can do so by enabling the template parameter (see file fitthesis.cls).
  \item You can generate a sample of the boards to which the work will be exported in the information system of the faculty. For the dissertation work, you can turn on the parameter in the template (see fitthesis.cls for more).
  \item Remember that you have to hand in source files as well as (both) PDF versions on a~CD or some other storage media attached to the technical report.
\end{enumerate}

Content of the work can be generated by command \tt \textbackslash tableofcontents \rm (included in the template). Attachments are included intentionally.

\subsection*{Instructions for double-sided print}
\begin{itemize}
\item \textbf{It is recommended to consult double-sided print with your supervisor.}
\item If your work is printed on both sides and its thickness is smaller than the thickness of the board itself, it does not look very appealing. 
\item It is enabled with a template parameter: \verb|\documentclass[twoside]{fitthesis}|
\item After the double-sided print, check whether the typesetting figure holds the same position for both sides of the paper. Lower-quality printers with duplex unit tend to shift the print by 1--3 mm. This may be solved by printing only the odd pages first and then putting them in the same tray and printing the even ones.
\item After the title page, table of contents, bibliography, list of attachments or any other lists, it is necessary to leave a blank page so the following section begins on the odd page (\textbackslash cleardoublepage).
\item The end result must be carefully checked.
\end{itemize}

\subsection*{Paragraph style}

Paragraphs are aligned to a block and there are several methods to format them. For paper literature, a frequent method to use is with the use of a paragraph stop, where the individual paragraphs of the text offset the first line of the paragraph with about one to two squares (always the same, the preselected value), that is about two widths of the upper letter M Of the basic text. The last line of the previous paragraph and the first line of the following paragraph are not separated by a vertical gap in this case. Break between these lines is the same as the break between the lines inside the paragraph. \cite{fitWeb} Another method is paragraph offset, which is common with the electronic text typeset. The first line of the paragraph is not indented, and a vertical space of 1/2 line is inserted between paragraphs. Both methods can be used in your work, but the latter is often more appropriate. Methods are not appropriate to combine.

One of the above methods is set in the template as the default, the other one can be selected by the template parameter \uv{\tt odsaz\rm }.

\subsection*{Useful tools}
\label{tools}

The following list does not enumerate all usable tools. In case you know of any other tool that is proven to work for you, do not hesitate to use it. Unless you know what tools to choose, you may consider using one of the following.

\begin{description}
   \item[\href{http://miktex.org/download}{MikTeX}] \LaTeX{} for Windows -- distribution with a simple installation and excellent automation of downloading packages.
   \item[\href{http://texstudio.sourceforge.net/}{TeXstudio}] Portable opensource GUI for \LaTeX{}. Ctrl+click allows you to switch between the source text and PDF. It comes with an integrated spell checker, syntax highlighter, etc. In order to use it, it is necessary to install MikTeX.
    \item[\href{http://www.winedt.com/}{WinEdt}] A good combination to use on Windows is WinEdt + MiKTeX. WinEdt is a GUI for Windows. In order to use it, you have to download \href{http://miktex.org/download}{MikTeX} or  \href{http://www.tug.org/texlive/}{TeX Live} first. 
  \item[\href{http://kile.sourceforge.net/}{Kile}] Editor for Desktop Environment KDE (Linux). Allows a live preview. In order to use it it is necessary to have \href{http://www.tug.org/texlive/}{TeX Live} and Okular installed. 
   \item[\href{http://jabref.sourceforge.net/download.php}{JabRef}] Nice and simple Java program for management of files with bibliography (literature). There is no need to learn anything new -- it provides a simple window and a~form for item editing.
   \item[\href{https://inkscape.org/en/download/}{InkScape}] Portable opensource editor for vector graphics (SVG and PDF). Excellent tool for creating images for scientific texts. It is hard to master, but the results are worth it.
   \item[\href{https://git-scm.com/}{GIT}] Excellent tool for collaboration on team projects, yet it also offers significant help to individuals. It provides simple version control, backups management and code portability.
   \item[\href{http://www.overleaf.com/}{Overleaf}] Online tool for \LaTeX{}. It provides a direct preview, simple cooperation (supervisor is able to continuously monitor the work progress), searching in the source texts, spell checking, etc. Although it is free, there are certain limitations in the free version (it may suffice for someone's dissertation thesis, though they can appear even when working on bachelor thesis) and it is slower for longer texts.
\end{description}

\subsection*{Useful packages for \LaTeX}

Students often encounter the same typesetting problems. Some of them can be solved with the following packages for \LaTeX:

\begin{itemize}
  \item \verb|amsmath| -- provides extended options for typesetting of equations,
  \item \verb|float, afterpage, placeins| -- image positioning adjustments,
  \item \verb|fancyvrb, alltt| -- editing of properties of the Verbatim environment,
  \item \verb|makecell| -- expands the options for tables,
  \item \verb|pdflscape, rotating| -- page rotation for 90 degrees (for images or tables),
  \item \verb|hyphenat| -- words splitting adjustments,
  \item \verb|picture, epic, eepic| -- direct image drawing.
\end{itemize}

Some of the packages are included directly in a template (see bottom part of the file fitthesis.cls). It may also help to see their documentation.

A fixed width column which is left aligned in a table is in the template defined as \uv{L} (it can be used as \uv{p}).

