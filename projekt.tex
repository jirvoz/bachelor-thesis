%==============================================================================
% tento soubor pouzijte jako zaklad
% this file should be used as a base for the thesis
% Autoři / Authors: 2008 Michal Bidlo, 2016 Jaroslav Dytrych
% Kontakt pro dotazy a připomínky: dytrych@fit.vutbr.cz
% Contact for questions and comments: dytrych@fit.vutbr.cz
%==============================================================================
% kodovani: UTF-8 (zmena prikazem iconv, recode nebo cstocs)
% encoding: UTF-8 (you can change it by command iconv, recode or cstocs)
%------------------------------------------------------------------------------
% zpracování / processing: make, make pdf, make clean
%==============================================================================
% Soubory, které je nutné upravit: / Files which have to be edited:
%   projekt-20-literatura-bibliography.bib - literatura / bibliography
%   projekt-01-kapitoly-chapters.tex - obsah práce / the thesis content
%   projekt-30-prilohy-appendices.tex - přílohy / appendices
%==============================================================================
\documentclass[english,zadani]{fitthesis} % bez zadání - pro začátek práce, aby nebyl problém s překladem
%\documentclass[english]{fitthesis} % without assignment - for the work start to avoid compilation problem
%\documentclass[zadani]{fitthesis} % odevzdani do wisu - odkazy jsou barevné
%\documentclass[english,zadani]{fitthesis} % for submission to the IS FIT - links are color
%\documentclass[zadani,print]{fitthesis} % pro tisk - odkazy jsou černé
%\documentclass[zadani,cprint]{fitthesis} % pro barevný tisk - odkazy jsou černé, znak VUT barevný
%\documentclass[english,zadani,print]{fitthesis} % for the color print - links are black
%\documentclass[english,zadani,cprint]{fitthesis} % for the print - links are black, logo is color
% * Je-li práce psaná v anglickém jazyce, je zapotřebí u třídy použít 
%   parametr english následovně:
%   If thesis is written in english, it is necessary to use 
%   parameter english as follows:
%      \documentclass[english]{fitthesis}
% * Je-li práce psaná ve slovenském jazyce, je zapotřebí u třídy použít 
%   parametr slovak následovně:
%   If the work is written in the Slovak language, it is necessary 
%   to use parameter slovak as follows:
%      \documentclass[slovak]{fitthesis}
% * Je-li práce psaná v anglickém jazyce se slovenským abstraktem apod., 
%   je zapotřebí u třídy použít parametry english a enslovak následovně:
%   If the work is written in English with the Slovak abstract, etc., 
%   it is necessary to use parameters english and enslovak as follows:
%      \documentclass[english,enslovak]{fitthesis}

% Základní balíčky jsou dole v souboru šablony fitthesis.cls
% Basic packages are at the bottom of template file fitthesis.cls
% zde můžeme vložit vlastní balíčky / you can place own packages here
\usepackage{listings}

% Kompilace po částech (rychlejší, ale v náhledu nemusí být vše aktuální)
% Compilation piecewise (faster, but not all parts in preview will be up-to-date)
% \usepackage{subfiles}

% Nastavení cesty k obrázkům
% Setting of a path to the pictures
\graphicspath{{obrazky-figures/}{./obrazky-figures/}}
%\graphicspath{{obrazky-figures/}{../obrazky-figures/}}

%---rm---------------
\renewcommand{\rmdefault}{lmr}%zavede Latin Modern Roman jako rm / set Latin Modern Roman as rm
%---sf---------------
\renewcommand{\sfdefault}{qhv}%zavede TeX Gyre Heros jako sf
%---tt------------
\renewcommand{\ttdefault}{pcr}% zavede Courier tt font

% vypne funkci šablony, která automaticky nahrazuje uvozovky,
% aby nebyly prováděny nevhodné náhrady v popisech API apod.
% disables function of the template which replaces quotation marks
% to avoid unnecessary replacements in the API descriptions etc.
\csdoublequotesoff

% =======================================================================
% balíček "hyperref" vytváří klikací odkazy v pdf, pokud tedy použijeme pdflatex
% problém je, že balíček hyperref musí být uveden jako poslední, takže nemůže
% být v šabloně
% "hyperref" package create clickable links in pdf if you are using pdflatex.
% Problem is that this package have to be introduced as the last one so it 
% can not be placed in the template file.
\ifWis
\ifx\pdfoutput\undefined % nejedeme pod pdflatexem / we are not using pdflatex
\else
  \usepackage{color}
  \usepackage[unicode,colorlinks,hyperindex,plainpages=false,pdftex]{hyperref}
  \definecolor{links}{rgb}{0.4,0.5,0}
  \definecolor{anchors}{rgb}{1,0,0}
  \def\AnchorColor{anchors}
  \def\LinkColor{links}
  \def\pdfBorderAttrs{/Border [0 0 0] }  % bez okrajů kolem odkazů / without margins around links
  \pdfcompresslevel=9
\fi
\else % pro tisk budou odkazy, na které se dá klikat, černé / for the print clickable links will be black
\ifx\pdfoutput\undefined % nejedeme pod pdflatexem / we are not using pdflatex
\else
  \usepackage{color}
  \usepackage[unicode,colorlinks,hyperindex,plainpages=false,pdftex,urlcolor=black,linkcolor=black,citecolor=black]{hyperref}
  \definecolor{links}{rgb}{0,0,0}
  \definecolor{anchors}{rgb}{0,0,0}
  \def\AnchorColor{anchors}
  \def\LinkColor{links}
  \def\pdfBorderAttrs{/Border [0 0 0] } % bez okrajů kolem odkazů / without margins around links
  \pdfcompresslevel=9
\fi
\fi
% Řešení problému, kdy klikací odkazy na obrázky vedou za obrázek
% This solves the problems with links which leads after the picture
\usepackage[all]{hypcap}

% Informace o práci/projektu / Information about the thesis
%---------------------------------------------------------------------------
\projectinfo{
  %Prace / Thesis
  project={BP},            %typ práce BP/SP/DP/DR  / thesis type (SP = term project)
  year={2019},             % rok odevzdání / year of submission
  date=\today,             % datum odevzdání / submission date
  %Nazev prace / thesis title
  title.cs={Výkonnostní testování plánovače linuxového kernelu},  % název práce v češtině či slovenštině (dle zadání) / thesis title in czech language (according to assignment)
  title.en={Performance Testing of Linux Kernel Scheduler}, % název práce v angličtině / thesis title in english
  title.length={13.2cm}, % nastavení délky bloku s titulkem pro úpravu zalomení řádku (lze definovat zde nebo níže) / setting the length of a block with a thesis title for adjusting a line break (can be defined here or below)
  %Autor / Author
  author.name={Jiří},   % jméno autora / author name
  author.surname={Vozár},   % příjmení autora / author surname 
  %author.title.p={Bc.}, % titul před jménem (nepovinné) / title before the name (optional)
  %author.title.a={Ph.D.}, % titul za jménem (nepovinné) / title after the name (optional)
  %Ustav / Department
  department={UITS}, % doplňte příslušnou zkratku dle ústavu na zadání: UPSY/UIFS/UITS/UPGM / fill in appropriate abbreviation of the department according to assignment: UPSY/UIFS/UITS/UPGM
  % Školitel / supervisor
  supervisor.name={Viktor},   % jméno školitele / supervisor name 
  supervisor.surname={Malík},   % příjmení školitele / supervisor surname
  supervisor.title.p={Ing.},   %titul před jménem (nepovinné) / title before the name (optional)
  %supervisor.title.a={},    %titul za jménem (nepovinné) / title after the name (optional)
  % Klíčová slova / keywords
  keywords.cs={Linux, jádro, plánovač úloh, CFS, testování, měření výkonu, vizualiace, strojové učení}, % klíčová slova v českém či slovenském jazyce / keywords in czech or slovak language
  keywords.en={Linux, kernel, task scheduler, CFS, testing, performance measurement, visualization, machine learning}, % klíčová slova v anglickém jazyce / keywords in english
  % Abstrakt / Abstract
  abstract.cs={Výkon plánovače procesů v jádře operačního systému značně
    ovlivňuje rychlost a odezvu všech aplikací, které na něm běží. Jakýkoli
    propad výkonu pak může mít kritické důsledky na běhu aplikací. S příchodem
    každé nové technologie (např. symetrický multiprocesing) se kód plánovače vyvíjí
    a rozšiřuje. Proto jsou potřeba regresní testy nejen na jeho fukčnost, ale i
    výkon. Tato práce mapuje metody testování plánovače operačního systému Linux
    ve firmě Red Hat. Popisuje způsoby měření výkonu plánovače, sbírání
    informací o jeho chování, ukládání sesbíraných dat a jejich vizualiaci.
    Hlavním cílem práce je pak návrh a implementace nového způsobu vizualizace
    dlouhodobých měření a využití strojového učení pro automatciké rozpoznání
    degradace výkonu mezi dvěma výsledky.}, % abstrakt v českém či slovenském jazyce / abstract in czech or slovak language
  abstract.en={Performance of process scheduler in a kernel of an operating
    system significantly influences throughput and latency of all applications
    running above it. Any performance drop can have critical consequences on the
    applications. With the arrival of every new technology (e.g. symetric
    multiprocesing) the code of the scheduler evolves and grows. This requires
    not only functional, but also performance regression testing. This work
    presents methods of performance testing used in the Red Hat, Inc. company.
    It describes how one can measure performance of the Linux process scheduler
    in the Linux kernel, collect statistics about its behavior, store the
    collected data, and visualize them. The goal of this work is to
    design and implement a new technique of visualization of long-term
    measurements and utilization of machine learning for automatic
    classification of performance degradation between different results.}, % abstrakt v anglickém jazyce / abstract in english
  %abstract.en={An abstract of the work in English will be written in this paragraph.},
  % Prohlášení (u anglicky psané práce anglicky, u slovensky psané práce slovensky) / Declaration (for thesis in english should be in english)
%   declaration={Prohlašuji, že jsem tuto bakalářskou práci vypracoval samostatně pod vedením pana X...
% Další informace mi poskytli...
% Uvedl jsem všechny literární prameny a publikace, ze kterých jsem čerpal.},
  declaration={Hereby I declare that this bachelor's thesis was prepared as an
    original author’s work under the supervision of Ing. Viktor Malík.
    The supplementary information was provided by Ing. Pavel Tišnovský, Ph.D.from
    Red Hat Czech s.r.o. and Ing. Tomáš Fiedor.
    All the relevant information sources, which were used during preparation of
    this thesis, are properly cited and included in the list of references.},
  % Poděkování (nepovinné, nejlépe v jazyce práce) / Acknowledgement (optional, ideally in the language of the thesis)
%   acknowledgment={V této sekci je možno uvést poděkování vedoucímu práce a těm, kteří poskytli odbornou pomoc
% (externí zadavatel, konzultant, apod.).},
  acknowledgment={I would like to thank to my supervisor Ing. Viktor Malík and
    consultants Ing. Pavel Tišnovský, Ph.D.from Red Hat Czech s.r.o. and Ing.
    Tomáš Fiedor for ...},
  % Rozšířený abstrakt (cca 3 normostrany) - lze definovat zde nebo níže / Extended abstract (approximately 3 standard pages) - can be defined here or below
  %extendedabstract={Do tohoto odstavce bude zapsán rozšířený výtah (abstrakt) práce v českém (slovenském) jazyce.},
  %faculty={FIT}, % FIT/FEKT/FSI/FA/FCH/FP/FAST/FAVU/USI/DEF
  faculty.cs={Fakulta informačních technologií}, % Fakulta v češtině - pro využití této položky výše zvolte fakultu DEF / Faculty in Czech - for use of this entry select DEF above
  faculty.en={Faculty of Information Technology}, % Fakulta v angličtině - pro využití této položky výše zvolte fakultu DEF / Faculty in English - for use of this entry select DEF above
  department.cs={Ústav matematiky}, % Ústav v češtině - pro využití této položky výše zvolte ústav DEF nebo jej zakomentujte / Department in Czech - for use of this entry select DEF above or comment it out
  department.en={Institute of Mathematics} % Ústav v angličtině - pro využití této položky výše zvolte ústav DEF nebo jej zakomentujte / Department in English - for use of this entry select DEF above or comment it out
}

% Rozšířený abstrakt (cca 3 normostrany) - lze definovat zde nebo výše / Extended abstract (approximately 3 standard pages) - can be defined here or above
\extendedabstract{Do tohoto odstavce bude zapsán rozšířený abstrakt práce v českém
  (slovenském) jazyce: úvod, popis vlastního řešení a shrnutí a zhodnocení
  dosažených výsledků.

  Plánovač procesů v operačním systému se stará o přidělování procesorového času
  běžícím procesům a jejich rovnoměrné rozložení mezi procesorové jádra. Výkon
  plánovače procesů pak silně ovlivňuje i výkon samotných aplikací běžících na
  daném operačním systému. Jakýkoli propad výkonu pak může mít v komerční sféře
  za následky vysoké finanční ztráty.

  Plánovač procesů funguje velmi jednoduše pro systémy s jedním jádrem, ovšem s
  příchodem vícejádrových procesorů se plánování zkomplikovalo vyvažováním front
  procesů mezi jádry a trvalo, než se všechny problémy vyladily. Nové komplikace
  plánování pak přinesly víceprocesorové systémy s neuniformní dobou přístupu do
  paměti. Plánování na této architektuře je stále ve vývoji a tím i náchylné k
  chybám způsobujícím propad výkonu. Proto je potřeba regresní testování výkonu
  nových verzí jádra pro včasné odhalení těchto chyb.

  Na rozdíl od funkčního testování není výkonové testování není jednoznačný
  výsledek, jestli test doběhl. Pro zjištění změny výkonu je potřeba relativní
  porovnání naměřených hodnot s referenčním výsledkem z předchozí verze nebo
  jiné konfigurace a určení prahu mezi odchylkou měření a skutečným propadem
  výkonu. Propady výkonu plánovače navíc nejsou způsobeny pomalým kódem, ale
  chybným přemisťováním procesů a jejich dat mezi jádry a fyzickými procesory.

  Tato práce popisuje výkonové testování plánovače procesů Red Hat Enterprise
  Linuxu ve společnosti Red Hat, Inc. Běžný způsob zjišťování výkonu plánovače
  je měření benchmarkem, který simuluje reálnou zátěž. Benchmarků je více s
  různými způsoby zátěže, především však pomocí většího množství procesů nebo
  vláken s vnitřní komunikací. Výsledky jsou pak systematicky ukládány pro
  snažší tvorbu porovnání výkonu. Pro efektivní analýzu výkonu je pak nutná
  volba správné vizualiace pro rychlé odhalení zdroje problému.

  Práce navrhuje a implementuje novou metodu zobrazování dlouhodobých výsledků
  měření výkonu nazvanou \emph{timelines}. Výstupem je pak HTML stránka
  obsahující krabicové (box plot) grafy a shrnující tabulky znázorňující změnu
  výkonu a přesnost měření v průběhu verzí jádra operačního systému. Tento
  výstup již pomohl s redukcí nestabilních částí benchmarků a průběžně slouží
  pro sledování změn výkonu a dohledávání verzí, kde výkon změnil.

  Dále práce navrhuje použití strojového učení pro automatcikou klasifikaci
  porovnání dvou výsledků, zda se mezi nimi i přes nepřesnost měření a šum
  projevuje propad výkonu. Práce navrhuje předzpracování výsledků měření pro
  učení klasifikátorů a porovnává různé klasifikátory pro budoucí začlenění do
  generátoru zpráv s porovnáním nových výsledků pro urychlení jejich analýzy.}
% nastavení délky bloku s titulkem pro úpravu zalomení řádku - lze definovat zde nebo výše / setting the length of a block with a thesis title for adjusting a line break - can be defined here or above
%\titlelength{14.5cm}


% řeší první/poslední řádek odstavce na předchozí/následující stránce
% solves first/last row of the paragraph on the previous/next page
\clubpenalty=10000
\widowpenalty=10000

\begin{document}
  % Vysazeni titulnich stran / Typesetting of the title pages
  % ----------------------------------------------
  \maketitle
  % Obsah
  % ----------------------------------------------
  \setlength{\parskip}{0pt}

  {\hypersetup{hidelinks}\tableofcontents}
  
  % Seznam obrazku a tabulek (pokud prace obsahuje velke mnozstvi obrazku, tak se to hodi)
  % List of figures and list of tables (if the thesis contains a lot of pictures, it is good)
  \ifczech
    \renewcommand\listfigurename{Seznam obrázků}
  \fi
  \ifslovak
    \renewcommand\listfigurename{Zoznam obrázkov}
  \fi
  % \listoffigures
  
  \ifczech
    \renewcommand\listtablename{Seznam tabulek}
  \fi
  \ifslovak
    \renewcommand\listtablename{Zoznam tabuliek}
  \fi
  % \listoftables 

  \ifODSAZ
    \setlength{\parskip}{0.5\bigskipamount}
  \else
    \setlength{\parskip}{0pt}
  \fi

  % vynechani stranky v oboustrannem rezimu
  % Skip the page in the two-sided mode
  \iftwoside
    \cleardoublepage
  \fi

  % Text prace / Thesis text
  % ----------------------------------------------
  \chapter{Introduction}
Performance of operating system is crucial, because it can significantly affect all the applications running above it.
When a regression in new version occurs, business applications can be even financially affected.

Scheduler was simple. Multi-core CPUs made the scheduler little more complex, but those complications were tuned through time.
Next milestone was introduction of multi-CPU machines with separated memory for each physical unit.

Compared to functional testing, performance isn't evaluated as true/false result, but relative change to previous measurement.
Due to this complexity is hard to use common tools for inspecting performance.
The biggest regression in scheduler doesn't hide in uneffective code, but in wrong placement of processes to cores and it's queues.

The usual testing method is simulating load simmilar to real usage.
To achieve this there are many benchmarks targetting different types of load, usually many parallel process, sometimes communicating between each other.

Good performance tests usually create great amount of data, which is required to be processed, aggregaed and visualised.

\chapter{Linux process scheduling}
Scheduler is a part of an operating system which assigns the processor time to tasks.
Its main goal is to maximize effectivity of the CPU usage and fairness of the
CPU time assigned to each task.

There are two opposing targets for a scheduler: either maximizing throughput or
minimizing latency. Lower amount of context switches leaves more CPU time for
tasks, but raises the response time on system events.
While user's workstation aims for a low response time, computational servers
require high throughput. Scheduler can be usually tuned to fit the intended
purpose.

In this chapter we describe basic behavior of Linux process scheduler. Then we
compare uniform and non-uniform memory access on multiprocessor architectures
and how scheduler handles them.

\section{Completely Fair Scheduler}
Completely Fair Scheduler is the current Linux process scheduler, which was
merged into the version 2.6.23 of the Linux kernel in 2007. Its author is Ingo
Molnár, who is the author of the previous \emph{O(1) scheduler} as well.

It features queuing tasks in a red-black tree structure ordered by time spent
running on the CPU so far. Red-black tree is a binary search
tree with self-balancing mechanism based on marking nodes with either red or
black color.
When the scheduler needs to choose the next task to run, it
takes the leftmost node with the lowest execution time. 

The time complexity of the CFS scheduling is O(log N). Taking the leftmost node
with the next task can be done in a constant time, but queuing the task again
requires O(log N) operations to insert it back into the red-black tree. Even
with a higher scheduling complexity, the CFS scheduler has a better fairness and
responsiveness than the previous O(1) scheduler, which used a simple queue to
choose the next task.
% CFS can also handle priority better: it requires only one queue instead of two
% for each of 140 priorities in Linux for O(1)

On multi-core systems, the scheduler uses a separate queue for each core. In
order to effectively use the processing power, the scheduler must regularly
balance those queues by moving processes from the most busy cores to idle ones.

When moving processes between cores, scheduler takes in the account a topology
of the system. Loosing data from caches after a migration can have a bigger
impact on performance than leaving the process on the busy core.

% Linux kernel book
CFS solves this problem by using scheduling domains. Scheduling domain is a set
of CPUs that should be balanced between themselves. CPUs in a scheduling domain
are divided into groups. Scheduler checks load of the whole groups to decide if
there is a need to migrate processes between them.

There are multiple levels of scheduling domains with different parameters such as
how often is the load difference checked or how big the load difference between
groups must be to migrate tasks to balance the queues.
The lowest level is between hyper-threaded logical cores where are almost no
losses of cached data and rebalancing can be done very often.
A higher level is between physical processor cores where  cache losses can have
bigger impact on the decision.
Above cores can be processor sockets on machines with multiple physical
processors with different access speed to different memory sections.

Scheduling domains are regularly rebalanced by going up from bottom of the
scheduling domain hierarchy and by checking balance of the groups on each level.

% https://lwn.net/Articles/764482/
\begin{figure}
  \centering
  \includegraphics[width=4cm]{obrazky-figures/sched-domains}
  \caption{Hiearchy of scheduling domains}
  \label{fig:sched-dom}
\end{figure}

\section{Scheduling on SMP systems}
Symetric mutiprocessing (SMP) is an architecture of computer with multiple
physical processors, that have single shared memory, access to all IO devices
and run on the same instance of operating system. This allows the machine to
offer more processing power with a little overhead with memory sharing. Each
processor has still its own high speed cache, but with memory sharing there must
be maintained \emph{cache coherence} -- the uniformity of data shared between
processors in their caches.

For the main shared memory there are two ways of accesing it from multiple
processors: uniform (UMA) and non-uniform (NUMA) memory access.

\subsection{Uniform memory access}
In UMA architecture all processors share one memory controller they use to
access shared memory. Each processor has the same speed of memory access and
latency. They share one common access route to memory which brings more
simplicity at the cost of lower bandwidth and speed.

This architecture is easier for the scheduler to balance processes between
the physical processors. The time to access the shared memory is same on all of
the cores and there is no need to move the process memory to any other place for
faster access.

\subsection{Non-uniform memory access}
The problem with low bandwidth tries to solve the NUMA architecture.
It groups physical processors or group of cores with their own separate memory
and bus for faster access. This significantly improves overall memory throughput
of the system with correct utilization.

Nodes are also have to be connected to each other to access memory of other
nodes. That is achieved with interconnecting buses or even controllers. Each
manufacturer has its own technology implementing the interconnection
technologies: Intel uses Ultra Path Interconnect which replaced its QuickPath
Interconnect from older machines, AMD uses Infinity Fabric supersetting older
HyperTransport.

On bigger machines with larger amount of processors not every two processors are
connected, but access the data through a path of connected ones. This can be
seen on Figure \ref{fig:proliant} showing 8 NUMA node machine with advanced
structure of interconnect buses and controllers.

%Best Practices When Deploying Linux on DL980 4AA3-6556ENW.pdf
\begin{figure}
  \centering
  \includegraphics[width=14cm]{obrazky-figures/proliant}
  \caption{Architecture of NUMA communication on HP ProLiant DL980}
  \label{fig:proliant}
\end{figure}

Consequence of this is different latency between nodes which must be taken into
account when balancing tasks between nodes. Difference of access latency between
NUMA nodes for machine from Figure \ref{fig:proliant} shows part of output from
command \texttt{numactl --hardware}:

\begin{minipage}{\linewidth}
\begin{verbatim}
node distances:
node 0 1 2 3 4 5 6 7
0: 10 12 17 17 19 19 19 19
1: 12 10 17 17 19 19 19 19
2: 17 17 10 12 19 19 19 19
3: 17 17 12 10 19 19 19 19
4: 19 19 19 19 10 12 17 17
5: 19 19 19 19 12 10 17 17
6: 19 19 19 19 17 17 10 12
7: 19 19 19 19 17 17 12 10
\end{verbatim}
\end{minipage}\\

Balancing tasks between NUMA nodes is difficult for scheduler taking into
account expensive memory movement or access to different nodes. With wrong
approach the performance of NUMA system can even drop below the performance of
similar UMA system\footnote{http://highscalability.com/blog/2013/5/30/google-finds-numa-up-to-20-slower-for-gmail-and-websearch.html}.

Balancing processes between NUMA nodes is still in active development, which
brings many changes both to better or sometimes worse performance.


\chapter{Performance measurement}
Performance testing is examination of the system behavior under workload and its
effectivity of resource usage. Many systems depend on the time they are able to
respond and this property affects the usability of th system.

Compared to functional testing, performance testing does not produce exact true
or false result. It produces set of numerical values which must be compared to
pressumed values or values from other version to make a conclusion.

Next step is inspection of behavior of the system to understand the measured
values and determine the causes of the difference from the expected values.

For evaluation of system performance are usually used benchmarks.
Benchmarks generate artificial load imitating the load in real environment.
While stressing the system they also measure the performance. The benchmarks
typically return a value representing the performance of the system. The value
is usually in the form of the time that the task needed to finish or amount of
the operations that the system could perform per a unit of time.

Effectivity of task scheduling and of their migration between processors affects
the amount of the tasks that the system can handle and the time that a task
spends before finishing.

Although the benchmarks generate values that are suitable for comparison, they
do not provide any more detailed information about how the system achieved the
measured performance and where could be some possible bottlenecks.

To get a better insight into the behavior of the system, there are many tools to
collect information about the system behavior. Useful informations about the
scheduler include assignment of tasks to the processor cores, the time that the
tasks spent out of the CPU in queues, load of each processor core, and the
location of memory of the processes on NUMA systems.
% TODO cite http://www.brendangregg.com/activebenchmarking.html

\section{Performance metrics}
throughput

\section{Benchmarks}
In this chapter we will list the benchmarks that are currently used to evaluate
performance of latest kernel versions. The benchmarks are usually based on real
applications used both in scientific and business environments.

The benchmarks run in many threads or processes and feature communication
between the instances. With bad distribution of processes and threads by the
scheduler rises their time of waiting in the queue and the performance of the
system goes down. On NUMA systems performance depends moreover on placement of
data to the memory.

\subsection{NAS Parallel Benchmarks}
NAS Parallel Benchmarks is a set of benchmarks focused on performance of highly
parallel computations on supercomputers. In addition to floating point
computations, it targets communication and data movement among computation
nodes. The performed algorithms are based on large scale computational fluid
dynamics at the Numerical Aerodynamic Simulation (NAS) Program which is based at
NASA Ames Research Center.

Benchmarks are written in Fortran-90 or in C language, beacause these were the most
commonly used programming languages in scientific parallel computing community
at the time when the benchmarks were created. They can be compiled with different classes of
problem sizes to suit machines with different amount of memory and of computational
power.

Main output value of the benchmark is throughput measured in units called Mop/s
(millions of operations per second) representing amount of floating-point
operations per unit of time.

The benchmark also offers a few parameters that can passed to the benchmark
before the execution to tweak properties of behavior. One of them is a number of
computation threads, which in a lower amount slowers the run time, but allows to
measure behavior of the system without full usage. Figure \ref{fig:nas} shows
throughput with different number of threads on machine with 24 phisical cores
and Hyper-threading.

Downside of this benchmark is it can only run with fixed dataset, but not for a
fixed time period. This constraint makes the runtime of the benchmark with less threads longer

\begin{figure}
  \centering
  \includegraphics[width=12cm]{obrazky-figures/nas}
  \caption{Example of Scalar Penta-diagonal solver results from NAS Parallel with
    different number of computational threads.}
  \label{fig:nas}
\end{figure}

\subsection{SPECjbb2005}
Java Business Benchmark behaves as a server-side Java application. It is primarily focused on measuring performance of Java implementation, but also reflects performance of operating system and cpu itself.
It models system of wholesale company as a multitier application. The benchmark
itself creates the load, measures the throughput and also generates simple
report in HTML and raw text formats.

The main output value is \emph{throughput} in units called \emph{SPECjbb2005
  bops} \footnote{Business operations per second}.
In case of using more JVM instances, there is a second unit called SPECjbb2005
bops/JVM representing average throughput of single JVM\footnote{Java virtual machine} instance.
Another collected metric is memory comsumption, which isn't that useful in
scheduler performance monitoring.

\subsection{LINPACK benchmark}
LINPACK Benchmark comes from LINPACK package, wich was used to solve systems of linear equations in single or double precision.

\subsection{Stream benchmark}

\section{Performance analysis tools}

\subsection{time}
Time is a simple command for measuring the time that an application spent running. The
most common numbers it gives out are the total of real time that the application needed
to finish, the time that it spent in the user mode, and the time spent in the kernel mode.

Many benchmarks provide the execution time themselves which can make this
utility look useless. However the intersting information is the difference between
the userspace and the kernel time and the total execution time, which is time that
the application spent out of the CPU waiting in a queue.

It can be confused with bash builtin command \texttt{time}, which provides similar
information, but the real command can provide more verbose information with
a possibility of custom formating of output. It can be usually called from
\texttt{/usr/bin/time}.

\subsection{ps}
Ps is Linux command, which is used to display about active processes. Its name
stands for processes status. It can provide various information obtainable from
the virtual files in \texttt{/proc} directory. The most common information
include their PID, time spent on processor, state of the process, used memory,
associated terminal, the command that started the process and more.

Especially useful are optional columns \texttt{PSR} and \texttt{NUMA}. They
show the number of processor and the number of NUMA node where the process is
running. Continuous monitoring of those values can provide view on migration of
the process during its run time. 

Output of the command can be filtered in many ways. By default it shows only
processes from the current user and the current terminal, but it can list all
the processes on the system. The listing can be filtered using parameters by
most of the columns of information it provides. The listing can be limited for
instance by specific terminal, effective user, children of a specified process
or PID to single intended process.

\subsection{mpstat}

\subsection{turbostat}
This tool provides measuring of CPU usage and mainly its frequency.

\subsection{perf}


\chapter{Storing the results}
Benchmarks usually generate long human-readable output in text or even HTML
format. This is useful when analyzing single report. In the output are details
of the test run itself, simple resource usage or success of result validation.
However, the amount of result starts to rise with repeated runs, different
amount of instances and new versions kernels.

For the comparison of performance results, it is usually enough the number
representing throughput or time of each benchmark run. Those numbers can be
preprocessed from the benchmark output files to a format more suitable for quick
accesing required data.

\section{XML files}
XML is markup language, that can store heterogeneous data in tree structure.
The tree structure can effectively represent the test scenario running each
benchmark operation with different parameters and multiple repetitions.

Another feature of XML format is human-readability offering quick insight to stored
data just with any text editor.
This comes with a disadvantage of redundant data in form of repeated names of
tags and attributes which often take more space than the data itself. Parsing of
the data also take considerable amount of CPU time prolonging the duration of
analysis.

Example of results stored in XML format:
\begin{verbatim}
<?xml version="1.0"?>
<BeakerRunResult>
 <TestResult>
  <NASResult benchmarkName="lu_C_x">
   <threads number="002">
    <result timeReal="293.764" timeOutOfCPU="10.789" mops="7016.79"/>
    <result timeReal="291.454" timeOutOfCPU="7.086" mops="7075.39"/>
    <result timeReal="290.403" timeOutOfCPU="7.844" mops="7096.21"/>
    <result timeReal="288.959" timeOutOfCPU="4.079" mops="7131.56"/>
    <result timeReal="289.642" timeOutOfCPU="4.644" mops="7118.24"/>
   </threads>
   <threads number="004">
    <result timeReal="153.149" timeOutOfCPU="12.688" mops="13462.15"/>
    <result timeReal="154.477" timeOutOfCPU="18.743" mops="13344.54"/>
...
\end{verbatim}
This example shows beginning of XML file with important values from one NAS
Parallel benchmark run scenario. The XML file starts with root element
\texttt{<BeakerRunResult>} and \texttt{<TestResult>} node which are wrapping
\texttt{<NASResult>} nodes representing results from each benchmark operation
from NAS Parallel benchmark suite. Each benchmark operation is run with
different amount of threads in few loops to lower the measurement inaccuracy.
Nodes of results with the same number of threads are wrapped in
\texttt{<threads>} node. All the values are stored as attributes of the
corresponding node.

\section{Database}


\chapter{Displaying the results}
\section{Heat maps}
Heat maps are three dimensional graphs which are using color as the third
dimension for values. This allows to plot two dependencies of the values
compared to line graphs, which must use multiple lines to plot the same data.
Heatmapy škálují pro velké data. 
Data jsou kvantizovány do bucketů.

In the Figure \ref{fig:heatmap} is heat map showing utilization of all CPU cores over
time under workload. The data were collected by mpstat utility and processed to
show sum of user and kernel space utilization of each core. Plotting those data
using line graph with a line for every core would be confusing even for this
relatively small amount of CPUs.

\begin{figure}
  \centering
  \includegraphics[width=14cm]{obrazky-figures/heatmap}
  \caption{Example of heat map showing CPU utilization over time. The machine
    with 24 logical CPUs is under workload from NAS Parallel benchmark running
    in 16 threads.}
  \label{fig:heatmap}
\end{figure}

Another use of heat map is shown in Figure \ref{fig:numa_heatmap}. It does not
show utilization of threads, but their location on which NUMA node they are.
This heat map shows the migration of threads between NUMA nodes and the expected
result is not the highest value, but minimum of color changes in each line. The
shown data comes from NAS Parallel benchmark, which was run with 16 and 24
threads in 5 loops. The heat map shows better scheduler behavior on the 16
threads run than on the 24 threads run.

\begin{figure}
  \centering
  \includegraphics[width=14cm]{obrazky-figures/numa_heatmap}
  \caption{Example of heat map showing thread migration between NUMA nodes. The
    expected result is not the highest value, but minimum of color changes in
    each line. The shown result comes from machine with 24 logical CPUs running
    NAS Parallel benchmark on 16 and 24 threads in 5 loops.}
  \label{fig:numa_heatmap}
\end{figure}

\section{Boxplots}
Boxplot is method for displaying statistical properties of data groups. The
usually displayed values are minimum, 1\textsuperscript{st} quartile, median,
3\textsuperscript{rd} quartile and maximum.
This feature is useful to visualize accuaracy and reliability of measurement.
Then it's easier to distinguish performance regression from noise.

\begin{figure}
  \centering
  \includegraphics[width=12cm]{obrazky-figures/boxplot}
  \caption{Example of boxplot showing Mops of NAS benchmark with different
    number of threads}
\end{figure}


\chapter{Timelines}
An common way to get performance report is to compare two results. Usually, they
are called baseline and target results. The comparison of two results allows to
write many details about the measurement and changes between versions. Those
details usually contain clues to the cause of possible performance change.

However, sometimes is not enough to compare just two versions and larger amount
of results over longer period of time can bring a new perspective. There is much
more visible difference between the deviation from measurement error and the
performance change. It is also easier to find the versions, where a
performance degradation occured and where was fixed.

With larger amount of data, there can also emerge performance drops, which
appeared continuously over longer period of time and couldn't be spotted,
because they were in tolerance due to deviation.

\section{Result storage}
Benchmark results are stored on filesystem in directories. Each result consists
of two XML files with information and preprocessed main data from the test run,
more files with larger complementary data from analysis tools and compressed
original outputs from benchmark. First XML file contains metadata from the test
run including time, machine hostname, kernel and OS version, benchmark name,
configuration of environment, which could affect the result and few other data.
The second XML file contains the important preprocessed data itself.

The program has to go throgh all these results to choose the right ones for the
following creation of graphs.

\section{Comparison rules}
For automatic report generation are essential rules, which will specify results,
that can be used and in which role. I chose \emph{regular expressions} to match
properties of results. Regular expressions offer broad possibilities to describe
shape of kernel version or just value of any evironment configuration. To filter
all builds of kernel 4.18 works simple pattern \texttt{kernel-4.18\textbackslash
..*}.

To store the rules are stored in XML file with same node naming as in XML file
with result properties. The first level of XML document contains three nodes
representing purpose of rules.
\begin{itemize}
  \item \textbf{Baseline rules} specify the first result in plotted set. Acts as
    main result others are compared to.
  \item \textbf{Target rules} define the results to be plotted.
  \item \textbf{Starting rules} are for case, when base result is not from
    target set of results and specifying first target result with regex would be
    hard.
\end{itemize}
Each of theese nodes then contain in them nodes with the rules.


\chapter{Automatic evaluation}
With every expansion of regular kernel testing rises the amount of produced
results. With more machines with different configurations, benchmarks with
different focus or baselines from different supported versions, number of results
can rise with every new kernel version even by hundrets. This tends to automate
the repetitive classification of results between pass and fail to leave results
without significant change and focus on the ones with performance regressions.

\section{Marking of results}
To teach automatic classificator is essential large amount of data. To reduce
time spent on marking of the passed and failed results I created a form in
report to quicken this process.

On machine for storing results is running simple server application for
commiting requests from the form. The server is written in Python built on top
of Flask framework. On request it finds given result and modifies its file with
marking for learning.

\begin{figure}
  \centering
  \includegraphics[width=12cm]{obrazky-figures/teaching_table}
  \caption{Form from HTML comparison report page to mark data for machine learning}
\end{figure}


%\chapter{Future work}


\chapter{Notes}
\begin{verbatim}
Linux scheduler
    CFS
    numa planning
        not easy for scheduler
        manual pinning option
            numactl
        group imbalance bug
    tune profiles
        focus on throughput or latency
Performance
    performance isn't simple pass/fail
    comparison of baseline and target kernel
    running just benchmark is waste of testing potential, it's good to collect more data
        http://www.brendangregg.com/activebenchmarking.html
    provisioning of machines with Beaker
    collecting system load data
        mpstat
            usage of every cpu
        numastat
        numatop
            usage of cpu and memory on each node
        ps
            psr and numa column
            cpu time
        time
            total, user and system time spent
        turbostat
        perf stat
        lstopo for hardware schema image
        free
    benchmarks
        specjbb2005
        specjvm2008
        nas parallel
        linpack benchmark
        stream (memory throughput)
        specjbb2015 - not stable
        hackbench - not tried yet
    scenarios
        variable number of benchmark instances
        with and without pinning processes to specific numa nodes
    result storing
        preprocessing to xml
        storing to database
        aggregating results
        computing statistical data
    plotting
        lstopo for plotting hardware topology
        bargraphs to show inaccuaracy
            operations per second (runtime of benchmark)
            time out of cpu
        detailed comparsion of two results
        timeline with many results over longer period of time
        heatmaps for core usage over time
            http://www.brendangregg.com/HeatMaps/utilization.html
        graph of process migration between numa nodes
Timelines
    analysis
    used technologies
    desing
    implementation
    output
Automatic regression detection
    motivation
    marking data
    preprocessing for classificator
    methods
        linear logistic regression
        decision trees
    evaluation
\end{verbatim}

\chapter{Specification}
\begin{enumerate}
\item Get acquainted with the existing methods for measuring performance of the Linux kernel scheduler and with means of storing of benchmarks results for further processing.
\item Study possible ways of processing these results with a focus on graphic interpretation and on methods for detection of performance degradation.
\item Design and implement a method for efficient graphic interpretation of long-term measurements.
\item Design and implement a method for automatic detection of performance regression.
\item Demonstrate the functionality of your implementation on at least two versions of the Linux kernel.
\item Evaluate the obtained results and discuss possibilities of further development of the project, especially of the automatic detection of performance regression.
\end{enumerate}

Literature:
\begin{itemize}
\item Lozi, Jean-Pierre, et al. "The Linux scheduler: a decade of wasted cores." Proceedings of the Eleventh European Conference on Computer Systems. ACM, 2016.
\item Daniel, P., and Cesati Marco. "Understanding the Linux kernel." (2007).
\item Bailey, David H., et al. "The NAS parallel benchmarks." The International Journal of Supercomputing Applications 5.3 (1991): 63-73.
\end{itemize}

  
  % Kompilace po částech (viz výše, nutno odkomentovat)
  % Compilation piecewise (see above, it is necessary to uncomment it)
  %\subfile{projekt-01-uvod-introduction}
  % ...
  %\subfile{chapters/projekt-05-conclusion}


  % Pouzita literatura / Bibliography
  % ----------------------------------------------
\ifslovak
  \makeatletter
  \def\@openbib@code{\addcontentsline{toc}{chapter}{Literatúra}}
  \makeatother
  \bibliographystyle{bib-styles/slovakiso}
\else
  \ifczech
    \makeatletter
    \def\@openbib@code{\addcontentsline{toc}{chapter}{Literatura}}
    \makeatother
    \bibliographystyle{bib-styles/czechiso}
  \else 
    \makeatletter
    \def\@openbib@code{\addcontentsline{toc}{chapter}{Bibliography}}
    \makeatother
    \bibliographystyle{bib-styles/englishiso}
  %  \bibliographystyle{alpha}
  \fi
\fi
  \begin{flushleft}
  \bibliography{projekt-20-literatura-bibliography}
  \end{flushleft}

  % vynechani stranky v oboustrannem rezimu
  % Skip the page in the two-sided mode
  \iftwoside
    \cleardoublepage
  \fi

  % Prilohy / Appendices
  % ---------------------------------------------
  \appendix
\ifczech
  \renewcommand{\appendixpagename}{Přílohy}
  \renewcommand{\appendixtocname}{Přílohy}
  \renewcommand{\appendixname}{Příloha}
\fi
\ifslovak
  \renewcommand{\appendixpagename}{Prílohy}
  \renewcommand{\appendixtocname}{Prílohy}
  \renewcommand{\appendixname}{Príloha}
\fi
%  \appendixpage

% vynechani stranky v oboustrannem rezimu
% Skip the page in the two-sided mode
%\iftwoside
%  \cleardoublepage
%\fi
  
\ifslovak
%  \section*{Zoznam príloh}
%  \addcontentsline{toc}{section}{Zoznam príloh}
\else
  \ifczech
%    \section*{Seznam příloh}
%    \addcontentsline{toc}{section}{Seznam příloh}
  \else
%    \section*{List of Appendices}
%    \addcontentsline{toc}{section}{List of Appendices}
  \fi
\fi
  \startcontents[chapters]
  \setlength{\parskip}{0pt}
  % seznam příloh / list of appendices
  % \printcontents[chapters]{l}{0}{\setcounter{tocdepth}{2}}
  
  \ifODSAZ
    \setlength{\parskip}{0.5\bigskipamount}
  \else
    \setlength{\parskip}{0pt}
  \fi
  
  % vynechani stranky v oboustrannem rezimu
  \iftwoside
    \cleardoublepage
  \fi
  
  % Přílohy / Appendices
  % Tento soubor nahraďte vlastním souborem s přílohami (nadpisy níže jsou pouze pro příklad)
% This file should be replaced with your file with an appendices (headings below are examples only)

% Umístění obsahu paměťového média do příloh je vhodné konzultovat s vedoucím
% Placing of table of contents of the memory media here should be consulted with a supervisor
%\chapter{Obsah přiloženého paměťového média}

%\chapter{Manuál} % manual

%\chapter{Konfigurační soubor} % Configuration file

%\chapter{RelaxNG Schéma konfiguračního souboru} % Scheme of RelaxNG configuration file

%\chapter{Plakát} % poster

\chapter{How to use this template}
\label{how}

This chapter describes the individual parts of the template, which is followed by instructions on how to use this template.

This version of the template is only temporary. The new one is going to be published by the end of the year 2017 and it will contain new instructions on how to properly use the template, obligatory instructions on how to work on bachelor's and master's thesis (recapitulation of instructions which are available online) and non-obligatory recommendations by some from the selected supervisors that are already available on-line (see links in literature file). The only files that will contain changes in the new version are \texttt{projekt-01-kapitoly-chapters.tex} and \texttt{projekt-30-prilohy-appendices.tex}. Content of each will be removed by every student and its content will be replaced by their own. The template is thus ready to be used even in the current version.\cite{fitWeb}

  
  % Kompilace po částech (viz výše, nutno odkomentovat)
  % Compilation piecewise (see above, it is necessary to uncomment it)
  %\subfile{projekt-30-prilohy-appendices}
  
\end{document}
